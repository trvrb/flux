\documentclass[11pt,oneside,letterpaper]{article}

% graphicx package, useful for including eps and pdf graphics
\usepackage{graphicx}
\DeclareGraphicsExtensions{.pdf,.png,.jpg}

% basic packages
\usepackage{color} 
\usepackage{parskip}
\usepackage{float}

% text layout
\usepackage{geometry}
\geometry{textwidth=15cm} % 15.25cm for single-space, 16.25cm for double-space
\geometry{textheight=22cm} % 22cm for single-space, 22.5cm for double-space

% helps to keep figures from being orphaned on a page by themselves
\renewcommand{\topfraction}{0.85}
\renewcommand{\textfraction}{0.1}

% bold the 'Figure #' in the caption and separate it with a period
% Captions will be left justified
\usepackage[labelfont=bf,labelsep=period,font=small]{caption}

% review layout with double-spacing
%\usepackage{setspace} 
%\doublespacing
%\captionsetup{labelfont=bf,labelsep=period,font=doublespacing}

% cite package, to clean up citations in the main text. Do not remove.
\usepackage{cite}
%\renewcommand\citeleft{(}
%\renewcommand\citeright{)}
%\renewcommand\citeform[1]{\textsl{#1}}

% Remove brackets from numbering in list of References
\renewcommand\refname{\large References}
\makeatletter
\renewcommand{\@biblabel}[1]{\quad#1.}
\makeatother

\usepackage{authblk}
\renewcommand\Authands{ \& }
\renewcommand\Authfont{\normalsize \bf}
\renewcommand\Affilfont{\small \normalfont}

% notation
\usepackage{amsmath}
\usepackage{amssymb}
\newcommand{\virus}{\mathbf{x}}						% virus coordinate
\newcommand{\serum}{\mathbf{y}}						% serum coordinate
\newcommand{\viruses}{\mathbf{X}}					% set of virus coordinates
\newcommand{\sera}{\mathbf{Y}}						% set of serum coordinates
\newcommand{\ve}{v}									% virus effect
\newcommand{\se}{s}									% serum effect
\newcommand{\ves}{\mathbf{v}}						% set of virus effects
\newcommand{\ses}{\mathbf{s}}						% set of serum effects
\newcommand{\point}{f_{\scriptscriptstyle \vert}}	% point likelihood
\newcommand{\threshold}{f_{\textstyle \lrcorner}}	% threshold likelihood
\newcommand{\interval}{f_{\sqcup}}					% interval likelihood
\newcommand{\mdssd}{\varphi}						% MDS standard deviation
\newcommand{\virussd}{\sigma_x}						% virus / diffusion standard deviation
\newcommand{\serumsd}{\sigma_y}						% serum standard deviation
\newcommand{\tree}{\tau}							% phylogeny
\newcommand{\vn}{n}									% number of viruses
\newcommand{\sn}{k}									% number of sera
\newcommand{\normal}{\mathcal{N}}					% normal distribution
\newcommand{\bwithin}{\beta_c}						% within clade drift coefficient
\newcommand{\bsister}{\beta_s}						% sister clade drift coefficient
\newcommand{\bother}{\beta_t}						% across clade drift coefficient
\newcommand{\driftclade}[1]{x_\mathrm{#1}}
\setlength{\arraycolsep}{2pt}
\newcommand{\smalltwomatrix}[2]{\scriptsize \Big( \begin{matrix} #1 \\ #2 \end{matrix} \Big)}				% pretty inline matrix 
\newcommand{\smallfourmatrix}[4]{\scriptsize \Big( \begin{matrix} #1 & #2 \\ #3 & #4 \end{matrix} \Big)}	% pretty inline matrix 
\newcommand{\twomatrix}[2]{\left( \begin{matrix} #1 \\ #2 \end{matrix} \right)}								% pretty inline matrix 
\newcommand{\fourmatrix}[4]{\left( \begin{matrix} #1 & #2 \\ #3 & #4 \end{matrix} \right)}					% pretty inline matrix 

% comments
\usepackage{ulem}
\definecolor{blue}{rgb}{0.324,0.609,0.708}
\definecolor{purple}{rgb}{0.459,0.109,0.538}
\def\tb#1#2{\sout{#1} \textcolor{purple}{#2}} 
\def\tbc#1{\textcolor{purple}{[#1]}}
\def\ms#1#2{\sout{#1} \textcolor{blue}{#2}} 
\def\msc#1{\textcolor{blue}{[#1]}}

%%% TITLE %%%
\title{\vspace{1.0cm} \LARGE \bf 
Integrating influenza antigenic dynamics with molecular evolution
}

\author[1]{Trevor Bedford}
\author[2,3,4]{Marc A. Suchard}
\author[5]{Philippe Lemey}
\author[1]{Gytis Dudas}
\author[6]{Victoria Gregory}
\author[6]{Alan J. Hay}
\author[6]{John W. McCauley}
\author[7]{Colin A. Russell}
\author[7,8]{Derek J. Smith}
\author[1,9]{Andrew Rambaut}

\affil[1]{Institute of Evolutionary Biology, University of Edinburgh, Edinburgh, UK}
\affil[2]{Department of Biomathematics, David Geffen School of Medicine at UCLA, University of California, Los Angeles CA, USA}
\affil[3]{Department of Human Genetics, David Geffen School of Medicine at UCLA, University of California, Los Angeles CA, USA}
\affil[4]{Department of Biostatistics, UCLA Fielding School of Public Health, University of California, Los Angeles CA, USA}
\affil[5]{Department of Microbiology and Immunology, Katholieke Universiteit Leuven, Leuven, Belgium}
\affil[6]{Division of Virology, MRC National Institute for Medical Research, Mill Hill, London, UK}
\affil[7]{Department of Zoology, University of Cambridge, Cambridge, UK}
\affil[8]{Department of Virology, Erasmus Medical Centre, Rotterdam, Netherlands}
\affil[9]{Fogarty International Center, National Institutes of Health, Bethesda, MD, USA}

\date{}

\begin{document}

\maketitle

\pagebreak

%%% ABSTRACT %%%
\section*{Abstract}

Influenza viruses undergo continual antigenic evolution allowing mutant viruses to evade immunity acquired by the host population to previous virus strains.
Antigenic phenotype is often assessed through pairwise measurement of cross-reactivity between influenza strains using the hemagglutination inhibition (HI) assay.
Here, we extend previous approaches to antigenic cartography, which seeks to place strains on an antigenic map, such that distances on this map best recapitulate titers observed across multiple HI assays.
In our model, we simultaneously characterize antigenic and genetic evolution by including an evolutionary model in which antigenic location diffuses over a shared virus phylogeny.
Using HI data for four lineages of influenza, encompassing influenza A subtypes H3N2 and H1N1, and influenza B lineages Victoria and Yamagata, we determine average rates of antigenic drift for each lineage, as well as year-to-year variability in the rate of drift.
Through comparison with epidemiological data, we demonstrate a year-to-year correlation between drift and incidence and present evidence that antigenic drift mediates interference between influenza lineages.
We investigate the selective underpinnings for differing antigenic dynamics across lineages and show that A/H3N2 benefits from both a higher influx of new antigenic mutations and also from more efficient conversion of antigenic variation into fixed differences.
This work does much to elucidate the antigenic dynamics of influenza lineages, but also allows for substantial future advances in investigating the dynamics of influenza and other antigenically-variable pathogens by providing a model that intimately combines molecular and antigenic evolution.

\pagebreak

%%% INTRODUCTION %%%
\section*{Introduction}

Seasonal influenza infects between 10\% and 20\% of the human population every year, causing an estimated 250,000 to 500,000 deaths annually \cite{flufactsheet}. 
While individuals develop long-lasting immunity to particular influenza strains after infection, antigenic mutations to the influenza virus genome result in proteins that are recognized to a lesser degree by the human immune system, leaving individuals susceptible to future infection. 
The influenza virus population continually evolves in antigenic phenotype in a process known as antigenic drift. 
A large proportion of the disease burden of influenza stems from antigenic drift; it is why vaccines remain only transiently effective. 
A thorough understanding of the process of antigenic drift is essential to public health efforts to control mortality and morbidity through the use of a seasonal influenza vaccine.

Before 2009, there were four major clades or lineages of influenza circulating within the human population: the H3N2 and H1N1 subtypes of influenza A, and the Victoria and Yamagata lineages of influenza B. 
In the case of influenza A, subtypes A/H3N2 and A/H1N1 refer to the genes, hemagglutinin (H or HA) and neuraminidase (N or NA), that are primarily responsible for the antigenic character of a strain. 
In the case of influenza B, Victoria (B/Vic) and Yamagata (B/Yam) refer to antigenically distinct lineages which diverged from a single lineage prior to 1980 \cite{Rota90}.
Mutations to the HA1 region of the hemagglutinin protein are thought to drive the majority of antigenic drift in the influenza virus \cite{Wiley81, Nelson07NatRevGenet}. 
Experimental characterization of antigenic phenotype is possible through the hemagglutination inhibition (HI) assay \cite{Hirst43}, which measures the cross-reactivity of one virus strain to serum raised against another strain through challenge or vaccination. 
Sera from older strains react poorly with more evolved viruses resulting in new strains having a transmission advantage over previously established strains.

The results of many HI assays across a multitude of virus strains of a single subtype can be combined to yield a two-dimensional map, quantifying antigenic similarity and distance \cite{Smith04}. 
The antigenic map of influenza A/H3N2 has shown largely linear movement of the influenza virus population since its introduction in 1968. 
Evolution of antigenic phenotype appears punctuated with episodes of more rapid innovation interspersed by periods of stasis, while genetic evolution appears more continuous \cite{Smith04}, suggesting that a relatively small number of genetic changes or combinations of genetic changes may drive changes in antigenic phenotype. 
The process of antigenic drift results in the rapid turnover of the virus population. 
Although mutation occurs rapidly, genetic diversity among contemporaneous viruses is low and phylogenetic analysis shows a characteristically `spindly' tree with a single predominant trunk lineage and transitory side branches that persist for only 1--5 years \cite{Fitch97}.

Previously, the antigenic and genetic patterns of influenza evolution have been analyzed essentially in isolation. 
An antigenic map is constructed from a panel of HI measurements, and a phylogenetic tree is constructed from sequence data. 
However, the opportunity for a combined approach exists as both the antigenic map and the phylogenetic tree often contain many of the same isolates. 
Here, we implement a flexible Bayesian approach to characterize jointly the antigenic and genetic evolution of the influenza virus population. 
We apply this approach to investigate the dynamics of A/H3N2, A/H1N1, B/Vic and B/Yam viruses, and, for the first time, present detailed reconstructions of the antigenic dynamics of all four circulating influenza clades.

%%% RESULTS %%%
\section*{Results and Discussion}

\subsection*{Antigenic and evolutionary cartography}

In order to assess patterns of antigenic evolution among influenza strains, we implemented a Bayesian probabilistic analog of multidimensional scaling (MDS), referred to here as BMDS (see Methods).
In this model, viruses and sera are given $N$-dimensional locations, thus specifying an `antigenic map', such that distances between viruses and sera in this space are inversely proportional to cross-reactivity.
In the BMDS model, a map distance of one antigenic unit translates to an expectation of a 2-fold drop in HI titer between virus and sera.
Maps that produce pairwise distances most congruent with the observed titers will have a high likelihood and will be favored by the BMDS model.
We integrate over sources of uncertainty, such as antigenic locations, in a flexible Bayesian fashion.
We apply this model to HI measurements of virus isolates against post-infection ferret antisera for influenza A/H3N2, A/H1N1, B/Vic and B/Yam (see Methods).

We test model performance by constructing training datasets representing 90\% of the HI measurements for each of the four influenza lineages and test datasets representing the remaining 10\% of the measurements for each lineage. 
By fitting the BMDS model to the training dataset, we are able to predict HI titers in the test dataset and compare these predicted titers to observed titers. 
We begin with Bayesian analogs of the models used by Smith et al.\ \cite{Smith04}, in which viruses and sera are represented as $N$-dimensional locations, expected titer is relative to the maximum titer of a particular ferret serum and virus and serum locations follow a diffuse prior.
After comparing models of differing dimensions, Smith et al.\ \cite{Smith04} arrive at a 2D model as the preferred model for their data.
We arrive at a similar conclusion, finding that a two dimensional model has better predictive power than models of lower of higher dimension in all four influenza lineages (models 1--5; Table \ref{errortable}).
We find that this 2D model performs well, yielding an average absolute predictive error of between 0.78 and 0.91 log$_2$ HI titers across influenza lineages (model 2; Table \ref{errortable}).
Consequently, we specify a two dimensional model in all subsequent analyses.
The finding of a low-dimensional map across influenza lineages extends previous studies in A/H3N2 \cite{Smith04} and remains an interesting and fundamental empirical observation.

%%% errortable %%%
\begin{table}[h]
	\centering
	\caption{\textbf{Average absolute prediction error of log$_2$ HI titer for test data across models and datasets.}}
	\label{errortable}	
	\makebox[\textwidth][c]{
	\begin{tabular}{ c c c c c c c c c } 
	\hline
	\multicolumn{5}{c}{} 	&	\multicolumn{4}{c}{Test error}  \\
	Model	&	Dimen 	& Location prior	& 	Serum effects 	&	Virus effects	& 	A/H3N2	& 	A/H1N1	& 	B/Vic	& 	B/Yam	\\
	\hline	
	1		&	1D 		& Uninformed		&	Fixed 			&	None			&	1.35	&	0.94 	&	0.90	&	1.08	\\
	2		&	2D 		& Uninformed		&	Fixed 			&	None			&	0.91 	&	0.78 	&	0.82	&	0.90	\\
	3		&	3D 		& Uninformed		&	Fixed 			&	None			&	0.93 	&	0.80 	&	0.85	&	0.92	\\
	4		&	4D 		& Uninformed		&	Fixed 			&	None			&	0.98 	&	0.84 	&	0.90	&	0.97	\\	
	5		&	5D 		& Uninformed		&	Fixed 			&	None			&	1.04 	&	0.89 	&	0.98	&	1.04	\\	
	6		&	2D 		& Diffusion/Drift	&	Fixed 			&	None			&	0.89	&	0.74	&	0.74	&	0.83	\\
	7		&	2D 		& Diffusion/Drift	&	Estimated 		&	None			&	0.77	&	0.73	&	0.66	&	0.75	\\	
	8		&	2D 		& Diffusion/Drift	&	Estimated		&	Estimated		&	0.76	&	0.71 	&	0.64	&	0.72	\\
	\hline
	\end{tabular}	
	}
\end{table}

Previous work on influenza antigenic and genetic evolution has examined antigenic and genetic patterns in isolation, and then compared the results to assess congruence of these two aspects of evolution \cite{Hay01, Smith04, Russell08}. 
Here, we simultaneously model antigenic and genetic evolution by adopting an evolutionary diffusion process \cite{Lemey10}, wherein a virus's antigenic character state evolves along branches of the phylogenetic tree according to a Brownian motion process (see Methods).
The phylogenetic diffusion process acts as a prior on virus locations, so that genetically similar viruses are expected to share similar antigenic locations.
The antigenic diffusion process includes both systematic drift along the tree and covariance induced by phylogenetic proximity.
We include sequence data for A/H3N2, A/H1N1, B/Vic and B/Yam to estimate this diffusion process, and find a small increase in predictive accuracy of between 0.02 and 0.08 log$_2$ HI titers when including the phylogenetic diffusion (model 6; Table \ref{errortable}).

We further extend the model by estimating the strength of overall reactivity of each serum rather than fixing this at its maximum titer, and additionally, by estimating the strength of reactivity of each virus isolate.
We refer to these estimates as serum effects and virus effects, respectively.
We find that including these effects decreases test error further, ranging in improvement from 0.03 to 0.13 log$_2$ HI titers (models 7 and 8; Table \ref{errortable}).

We find that the average absolute error in predicted log$_2$ HI titer is nearly constant with antigenic distance (Figure~\ref{error_by_distance_and_year}A), thus supporting our model assumption that the drop in log$_2$ titer is proportional to the Euclidean distance separating viruses and sera on the antigenic map.
Additionally, we find that the absolute error in predicted titer is nearly constant with time (Figure~\ref{error_by_distance_and_year}B).
Antigenic locations inferred by the model are well resolved; estimates of antigenic distance between pairs of viruses show relatively little variation across the posterior.
We estimate that virus distances have, on average, a 50\% credible interval of $\pm0.45$ antigenic units for A/H3N2, $\pm0.57$ units for A/H1N1, $\pm0.76$ units for B/Vic, and $\pm0.65$ units for B/Yam.

\subsection*{Antigenic evolution across influenza lineages}

Through our analysis, we reveal the antigenic, as well as evolutionary, relationships among viruses in influenza A/H3N2, A/H1N1, B/Vic and B/Yam, quantifying both antigenic and evolutionary distances between strains (Figure~\ref{map}).
Over the time period of 1968 to 2011, influenza A/H3N2 shows substantially more antigenic evolution than is exhibited by A/H1N1 over the course of 1977 to 2009 or B/Vic and B/Yam over the course of 1986 to 2011.
We observe prominent antigenic clusters in A/H3N2 and A/H1N1, but less prominent, though still apparent, clustering in B/Vic and B/Yam.
Antigenic clusters show high genetic similarity, so that we observe very few mutation events leading to each cluster, rather than the repeated emergence of clusters.
This analysis makes the fate of antigenic clusters obvious, with two clusters in A/H3N2 (Victoria/75 and Beijing/89) appearing to be evolutionary dead-ends.
Labeling of prominent antigenic clusters in figures \ref{map} and \ref{drift} is intended as a rough guide for orientation and not as exhaustive catalog of antigenic variation.

%%% map %%%
\begin{figure}[h]
	\centering		
	\includegraphics[width=1.0\textwidth]{figures/map}
	\caption{\textbf{Antigenic locations of A/H3N2, A/H1N1, B/Vic and B/Yam viruses showing evolutionary relationships between virus samples.} 
	Circles represent a posterior sample of virus locations and have been shaded based on year of isolation.
	Antigenic units represent two-fold dilutions of the HI assay.
	Absolute positioning of lineages, e.g.\ A/H3N2 and A/H1N1, is arbitrary.
	Lines represent mean posterior diffusion paths when virus locations are fixed.
	Prominent antigenic clusters are labeled after vaccine strains present within clusters, and are abbreviated from Hong Kong/68, England/72, Victoria/75, Bangkok/79, Sichuan/87, Beijing/89, Beijing/92, Wuhan/95, Sydney/97, Fujian/02, California/04, Wisconsin/05, Perth/09 (A/H3N2), USSR/77, Singapore/86, Beijing/95, New Caledonia/99, Solomon Islands/06 (H1N1), Victoria/87, Hong Kong/01, Malaysia/04, Brisbane/08 (Vic), Yamagata/88, Shanghai/02, Florida/06, Wisconsin/10 (Yam).} 
	\label{map} 
\end{figure}

HI assays lack sensitivity beyond a certain point, so that for A/H3N2, cross-reactive measurements only exist between strains sampled at most 11 years apart, leaving only threshold titers, e.g.\ `$<$40', in more temporally distant comparisons.  
Because of the threshold of sensitivity of the HI assay, it's impossible to distinguish a linear trajectory in 2D antigenic space from a slightly curved trajectory, so long as the curve does not bring antigenic phenotype full circle to have cross-reactive measurements between temporally distant strains.
To solve this problem of identifiability, we assumed a weak prior that favors linear movement in the 2D antigenic space (present in models 2 through 8; Table \ref{errortable}), with the slope of the linear relationship and the precision of the relationship incorporated into the Bayesian model (see Methods).
Because of this, we interpret map locations locally rather than globally.
We can determine rates of antigenic movement without knowing the larger configuration under which the movement occurs.
 
We find that influenza A/H3N2 evolved along antigenic dimension 1 at a rate of 1.01 antigenic units per year, with a 95\% highest posterior density (HPD) interval of 0.98--1.04 (Figure~\ref{drift}).
However, we observe occasional large jumps in antigenic phenotype (Figure~\ref{drift}), corresponding to cluster transitions identified by Smith et al. \cite{Smith04}.  
Most variation is contained within the first antigenic dimension, but dimension 2 occasionally shows variation when two antigenically distinct lineages emerge and transiently coexist (Figure~\ref{map}), as is the case with the previously identified Beijing/89 and Beijing/92 clusters.

%%% drift %%%
\begin{figure}[h]
	\centering		
	\includegraphics[width=0.75\textwidth]{figures/drift}
	\caption{\textbf{Antigenic drift of A/H3N2, A/H1N1, B/Vic and B/Yam viruses showing evolutionary relationships between virus samples.} 
	Antigenic drift is shown in terms of change of location in the first antigenic dimension through time.
	Circles represent a posterior sample of virus locations and have been shaded based on year of isolation.
	Antigenic units represent two-fold dilutions of the HI assay.
	Relative positioning of lineages, e.g.\ A/H3N2 and A/H1N1, in the vertical axis is arbitrary.
	Lines represent mean posterior diffusion paths when virus locations are fixed.
	Prominent antigenic clusters are labeled after vaccine strains present within clusters, and are abbreviated from Hong Kong/68, England/72, Victoria/75, Bangkok/79, Sichuan/87, Beijing/89, Beijing/92, Wuhan/95, Sydney/97, Fujian/02, California/04, Wisconsin/05, Perth/09 (A/H3N2), USSR/77, Singapore/86, Beijing/95, New Caledonia/99, Solomon Islands/06 (H1N1), Victoria/87, Hong Kong/01, Malaysia/04, Brisbane/08 (Vic), Yamagata/88, Shanghai/02, Florida/06, Wisconsin/10 (Yam).} 
	\label{drift} 
\end{figure}

We find that other clades of influenza evolved in antigenic phenotype substantially slower than A/H3N2 (Figure~\ref{drift}).
Influenza A/H1N1 evolved at a rate of 0.62 units per year (HPD 0.56--0.67), but showing a similar pattern of punctuated antigenic evolution with occasional larger jumps in phenotype, such as the emergence of the Solomon Islands/06 cluster.  
Influenza B/Victoria also evolved relatively slowly, with an average rates of 0.42 (HPD 0.32--0.51) units per year.
Influenza B/Vic appears to show some degree of punctuated antigenic evolution with a recent transition creating the Brisbane/08 cluster.
Interestingly, a minor lineage of B/Vic has persisted through 2011, appearing antigenically distinct from other 2011 viruses (Figure~\ref{drift}), though the eventual fate of this lineage has yet to be determined.
Influenza B/Yamagata evolved slower still, with an average rate of 0.32 (HPD 0.25--0.39) units per year.
Little punctuated evolution is obvious in the evolution of B/Yam.

Interestingly, although antigenic clusters seem to emerge in a punctuated fashion in influenza A/H3N2, year-to-year antigenic drift in the mean antigenic phenotype of the virus population is relatively constant through time with most years exhibiting an intermediate level of drift (Figure~\ref{jumps}).
We calculate year-to-year antigenic drift for years 1992 to 2011 by calculating the average location along dimension 1 of phylogenetic lineages present in the tree at year $i$ and comparing this location to the average location of phylogenetic lineages present in the tree at year $i-1$.
For A/H3N2, the largest cluster transition we observe is the emergence of the Sydney/97 cluster, evident as the appearance in 1997 of a large gap in virus locations along antigenic dimension 1 (Figure~\ref{drift}).
However, in 1997 only three of the nine viruses sampled show the Sydney/97 phenotype, while the remaining six viruses exhibit the old Wuhan/95 phenotype.
In 1998, three of the four viruses sampled show the Sydney/97 phenotype, with the remaining virus still showing the Wuhan/95 phenotype.
In 1999, all sampled viruses exhibit the Sydney/97 phenotype.
Thus, we observe fairly equal rates of year-to-year drift in mean antigenic phenotype between 1996 and 1999, rather than all drift in mean antigenic phenotype coinciding with the appearence of a new antigenic variant in 1997 (Figure~\ref{jumps}).
It is interesting that this phenomenon occurs despite the fact that antigenic drift is accomplished, not by steady movement of the entire virus population, but by the successive emergence of drifted variants that sweep through the virus population.
Because drifted variants require, on average, 2--3 years to take over the virus population, rates of year-to-year drift do not always spike heavily at the appearance of a new variant.

%%% jumps %%%
\begin{figure}[h]
	\centering		
	\includegraphics[width=0.95\textwidth]{figures/jumps}
	\caption{\textbf{Year-to-year antigenic drift between 1992 and 2009 in A/H3N2, A/H1N1, B/Vic and B/Yam viruses.} 
	(A) Time series of year-to-year antigenic drift across influenza clades from 1992 to 2011.
	Year-to-year antigenic drift for a virus clade for year $i$ is measured as the mean of antigenic dimension 1 of phylogenetic lineages in year $i$ compared to the mean of antigenic dimension 1 of phylogenetic lineages from the previous year $i-1$.
	For example, 2000 represents difference in antigenic dimension 1 between 1999 and 2000.
	Error bars represent 50\% credible intervals across posterior samples of year-to-year drift.
	(B) Histograms of year-to-year antigenic drift across influenza clades summarizing observations from 1992 to 2011.
	} 
	\label{jumps} 
\end{figure}

\subsection*{Epidemiological consequences of antigenic drift}

We investigate the relationship between rates of antigenic drift and USA incidence in A/H3N2, A/H1N1, B/Vic and B/Yam.
We take estimates of the rate of antigenic drift from the preceding section and compare these rates to measurements of relative incidence for each influenza clade calculated from the proportion of influenza viruses attributable to each clade (see Methods).
We analyze incidence from the 1998/1999 to the 2008/2009 seasons to avoid possible complications from the 2009 pandemic, finding that A/H3N2 accounted for 56\%, A/H1N1 for 20\%, B/Vic for 10\% and B/Yam for 14\% of viruses detected in the USA during this time period.
We find a significant correlation between rate of antigenic drift and relative incidence across the four clades  (Pearson correlation, $r=0.96$, $p = 0.021$).

We follow-up this analysis with a more detailed analysis of year-to-year variation in antigenic drift and clade-specific incidence in the USA.
We measure year-to-year antigenic drift within each clade as described in the previous section and shown in Figure \ref{jumps}.
We measure seasonal incidence of each clade by taking average influenza-like illness (ILI) percentage and multiplying this by the proportion of viruses attributable to a clade for each season.
This measure of incidence has previously been shown to have have predictive power in the analysis of seasonal influenza trends \cite{Goldstein11}.
In correlating antigenic drift to seasonal incidence we standardize this measure of incidence across seasons and across clades to measure effects in terms of standard deviations of incidence.
We find that years with high levels of antigenic drift coming into an influenza season, e.g.\ drift in A/H3N2 from 2000 to 2001, show high levels of incidence in that season, e.g.\ incidence of A/H3N2 in the 2001/2002 season (Figure~\ref{incidence}A). 
Here, within-clade drift explains 35\% of the variance in standardized incidence and represents a significant correlation (Pearson correlation, $r=0.60$, $p=1.6 \times 10^{-5}$).

We further show that incidence within a clade decreases with antigenic drift in other clades, e.g.\ drift in A/H1N1, B/Vic and B/Yam from 2000 to 2001 decrease incidence of A/H3N2 in the 2001/2002 season (Figure~\ref{incidence}B).
This also represents a significant relationship (Pearson correlation, $r=-0.20$, $p=0.024$).
We fit a linear model that predicts standardized incidence based upon drift within a clade, drift in other clades of the same virus type, and drift in clades of differing type (Table~\ref{interferencetable}).
For example, standardized incidence in A/H3N2 would be explained by drift within A/H3N2, drift in A/H1N1 and by drift across types in B/Vic and B/Yam.
We compared adjusted $R^2$ across models with different regression coefficients included, and arrived at a best-fitting model that explains 46\% of the variance in standardized incidence and includes the terms $\bwithin$, $\bsister$ and $\bother$, but does not include an intercept term.
We find that 1 unit of antigenic drift within a clade increases incidence by 1.26 standard deviations, while 1 unit of drift in a clade of the same type decreases incidence by 0.61 standard deviations and 1 unit of drift in clades of differing type decreases incidence by 0.32 standard deviations.

%%% interferencetable %%%
\begin{table}[h]
	\centering
	\caption{\textbf{Estimates of regression coefficients for the relationship of clade-specific USA incidence to antigenic drift within and across clades, where standardized incidence in A/H3N2 is predicted by $\bwithin \, \driftclade{H3} + \bsister \, \driftclade{H1} + \bother \, \driftclade{Vic} + \bother \, \driftclade{Yam}$ and standardized incidence in other clades follows an analogous relationship.}}
	\label{interferencetable}	
	\begin{tabular}{ c c c c } 
	\hline
	Parameter	&	Estimate	&	Standard error 	& 	$p$-value \\
	\hline		
	$\bwithin$	&	$1.26$		&	0.22			&	$1.1 \times 10^{-6}$ \\
	$\bsister$	&	$-0.61$		&	0.22			&	$0.008$ \\
	$\bother$	&	$-0.32$		&	0.11			&	$0.005$ \\
	\hline
	\end{tabular}
\end{table}

These findings clearly demonstrate the epidemiological importance of antigenic drift and show that incidence across  A/H3N2, A/H1N1, B/Vic and B/Yam is strongly influenced by levels of antigenic drift within each clade, consistent with antigenic drift increasing the susceptible pool available to a clade.
The finding of a negative correlation between incidence within one clade and antigenic drift in other clades supports the hypothesis that epidemiological interference exists between clades \cite{Ferguson03, Goldstein11}.
Interestingly, we find that interference between sister clades, e.g.\ B/Vic and B/Yam, appears stronger than interference between more divergent clades, e.g.\ B/Vic and A/H3N2.
If interference is mediated by CTL epitopes, then we would expect interference to decrease with evolutionary distance \cite{Voeten00}.

Furthermore, these findings suggest that the predominant clade of a particular influenza season could be estimated ahead of time by examining the degree of variation in antigenic drift between influenza clades.
However, this prediction would not be perfect, and 56\% of the variance in incidence remains unexplained by these estimates of antigenic drift.
Some of this variance may be explained by incorporating geographic variation in year-to-year antigenic drift; although there exists significant geographic variation in clade-specific incidence \cite{Finkelman07}, we calculated antigenic drift from global samples, rather than USA-specific samples.

\subsection*{Conversion of antigenic mutation into fixed differences}

We observe a faster rate of antigenic drift in influenza A/H3N2 than in A/H1N1 or in either lineage of influenza B.
Previous work using general epidemiological models has suggested that rates of antigenic drift may be influenced by both the fundamental reproductive number $R_0$ and the rate at which mutation decreases cross-immunity \cite{Gog02,Lin03}.
Owing to these and similar findings, subsequent work ascribed differences in the evolution of A/H3N2 relative to A/H1N1 and influenza B to either greater $R_0$ or greater mutation rate \cite{Ferguson03,Bedford12}.
Here, we examine the conversion of recently acquired antigenic mutation into fixed differences by comparing rates of antigenic diffusion to the degree of persistence on phylogeny branches in A/H3N2, A/H1N1, B/Vic and B/Yam (Table~\ref{diffusiontable}).
To facilitate comparisons, we measure the diffusion coefficient $D$ \cite{Pybus12} characterizing the antigenic movement of each influenza lineage (see Supp.\ Methods).
This coefficient is measured in terms of units$^2$/year and can be used to determine the diffusion length scale, that is the expected relationship between time and distance traveled, which is $2 \sqrt{t D}$ for a two-dimensional diffusion.
Higher diffusion coefficients mean a lineage is diffusing faster through antigenic space.
We compare diffusion coefficients between trunk branches and side branches, where trunk branches are defined as branches ancestral to the most recent virus sample and side branches are defined as those branches not ancestral to the most recent virus sample.

%%% diffusiontable %%%
\begin{table}[h]
	\centering
	\caption{\textbf{Estimates of fundamental diffusion coefficient $D$ across influenza lineages and phylogenetic branches, including posterior means and 50\% highest posterior density intervals.}}
	\label{diffusiontable}	
	\begin{tabular}{ c c c c } 
	\hline
	Lineage	&	Trunk branch $D$ 	& 	Side branch $D$		& 	Ratio \\
	\hline		
	A/H3N2	&	0.90 (0.79--0.97)	&	0.62 (0.57--0.65)	&	1.45 (1.31--1.57) \\
	A/H1N1	&	0.62 (0.50--0.71)	&	0.44 (0.38--0.48)	&	1.45 (1.18--1.70) \\
	B/Vic	&	0.63 (0.51--0.71)	&	0.62 (0.54--0.68)	&	1.02 (0.87--1.14) \\
	B/Yam	&	0.35 (0.29--0.40)	&	0.33 (0.28--0.36)	&	1.09 (0.93--1.23) \\
	\hline
	\end{tabular}	
\end{table}

Mutations must proceed through the sieve of natural selection, first appearing as transient differences restricted to side branches, while advantageous mutations will preferentially fix in the virus population and thus accrue on the trunk of the phylogeny.
We expect the diffusion coefficient on side branches to more closely resemble the underlying rate of antigenic mutation, before selection can strongly influence observed rates.
Here, we find that A/H3N2 and B/Vic have similar diffusion coefficients on side branches, which are larger than the diffusion coefficients of A/H1N1 and B/Yam (Table~\ref{diffusiontable}).
Thus, we expect that new antigenic mutations may occur at a faster rate in A/H3N2 and B/Vic than in A/H1N1 or B/Yam.
We find that trunk branches in both A/H3N2 and A/H1N1 evolve in antigenic phenotype faster than side branches (Table~\ref{diffusiontable}), suggesting that positive selection is promoting antigenic movement in these lineages.
Interestingly, B/Vic and B/Yam show less of a signature of positive selection, with similar diffusion coefficients on trunk and side branches (Table~\ref{diffusiontable}).
Thus, although drift forward along dimension 1 is driven by selection in all lineages (Figure~\ref{drift}), raw antigenic diffusion appears to be selected for more strongly in influenza A than in influenza B, giving influenza A an advantage in overall rate of antigenic drift.
Additionally, it appears that A/H3N2 and B/Vic benefit from a greater influx of new antigenic mutations, giving them a similar advantage over their sister lineages A/H1N1 and B/Yam, respectively.

In this context, it is especially interesting that adaptive evolution in the HA1 protein determined through analysis of nonsynonymous and synonymous substitutions is substantially stronger in A/H3N2 than in A/H1N1 \cite{Wolf06}; H3 appears to accumulate adaptive fixations at $\sim$150\% the rate of H1 \cite{Bhatt11}.
Here, we observe that A/H3N2 has a diffusion coefficient $\sim$141\% greater than A/H1N1 along side branches of the phylogeny.
These results suggest that the greater rate of adaptive fixation in A/H3N2 may be simply due to a greater abundance of new antigenic mutations to fix.

We suggest that the larger signal of positive selection for antigenic diffusion in influenza A relative to influenza B could be due to either a greater fundamental reproductive number $R_0$ or to pleiotropic effects associated with antigenic change.
In the latter case, we expect that many antigenic variants may be at an inherent disadvantage against other strains from decreased protein function \cite{Kaverin04,Rudneva12} or other effects, even though they benefit from a transmission advantage mediated by host immunity.
These mutations may not be promoted by natural selection to the same extent as antigenic variants that maintain functional equivalence.
Additionally, other work has shown strong pleiotropic constraints in the evolution of oseltamivir resistance in A/H1N1 \cite{Bloom10}.
Still, it is clear that more work is necessary to fully clarify the mechanistic underpinnings of the relative evolutionary success of different influenza lineages.

%%% CONCLUSIONS %%%
\section*{Conclusions}

In this study we provide a foundation for evolutionary antigenic cartography, which seeks to simultaneously assess antigenic phenotype and antigenic evolution.
We use this approach to characterize competitive dynamics across influenza clades A/H3N2, A/H1N1, B/Vic and B/Yam and show that antigenic evolution drives strain replacement in each clade and interference between clades.
We find that varying levels of mutational input and varying efficiencies of the conversion of antigenic polymorphism into fixed differences results in different rates of antigenic drift across influenza clades.
Influenza A/H3N2 benefits from a higher rate of new antigenic mutation coupled with strong adaptive evolution and evolves in antigenic phenotype faster than A/H1N1, B/Vic and B/Yam.
Correspondingly, we observe substantially greater levels of incidence in A/H3N2 than in other influenza clades.
We suggest that antigenic evolution strongly influences competitive dynamics both within and between influenza clades.

The statistical framework presented here represents a baseline to which further advancements in modeling antigenic phenotype and evolution may be made.
For example, our likelihood-based model facilitates the inclusion of possible covariates affecting immunological titer, which could include experimental factors such as red blood cell type used in the HI assay \cite{Lin12} and whether oseltamivir is included in the HI reaction \cite{Lin10}.
Additionally, this framework should be ideally suited to uncovering genetic determinants of antigenic change, as both the sequence state and antigenic location of internal nodes in the phylogeny may be estimated.
In this fashion, it should be possible to correlate sequence substitutions directly to antigenic diffusion.
Here, methods to efficiently calculate branch- and site-specific codon substitutions may be necessary to easily correlate sequence substitutions with antigenic change \cite{Lemey12}.

A deeper understanding of the process of antigenic evolution in the human influenza virus may eventually prove to be crucial in improving vaccine strain selection.

%%% METHODS %%%
\section*{Methods}

\subsection*{Methods summary}

We compiled an antigenic dataset of hemagglutination inhibition (HI) measurements of virus isolates against post-infection ferret sera for influenza A/H3N2 by collecting data from previous publications and WHO vaccine strain selection reports (see Supp.\ Text).
Hemagglutinin sequences were compiled from the Influenza Research Database \cite{IRD} and the EpiFlu Database \cite{GISAID}.
Antigenic and genetic data were subsampled to give similar quantities of data across years of the analysis.
Data gathering procedures resulted in 402 virus strains A/H3N2, 115 virus strains for A/H1N1, 179 virus strains for B/Vic and 174 virus strains for B/Yam.
Surveillance data was obtained from the Centers of Disease Control and Prevention FluView Influenza Reports from the yearly summaries of influenza seasons 1997--1998 to 2010--2011 \cite{CDCReports}.

HI titers were modeled by positioning viruses and sera in a $N$-d antigenic landscape, in which distances between viruses and sera yield an expectation for log$_2$ HI titer (see Supp.\ Text).
Thus, given a set of virus and serum locations we calculate the likelihood of observing a set of HI titers.
We initially begin with uniformed priors on virus and serum locations, but proceed to relax this assumption and incorporate priors informed by time and phylogenetic distance.
Here, we assume that viruses diffuse in antigenic space following a Brownian motion process, so that evolutionary similar viruses are expected to have similar antigenic locations.
We integrate across virus and serum locations and other parameters through the Markov chain Monte Carlo (MCMC) procedures implemented in the software package BEAST \cite{BEAST17}.

\subsection*{Availability}

Source code implementing the cartographic models has been made fully available as part of the software package BEAST \cite{BEAST17}, and can be downloaded from its Google code repository (http://code.google.com/p/beast-mcmc/).
Antigenic, genetic and surveillance data is available as supplemental information.

%%% ACKNOWLEDGMENTS %%%
\subsection*{Acknowledgments} 

We thank Richard Reeve, Dan Haydon, and Simon Frost for insights on antigenic modeling and MDS.
We acknowledge the laboratories that provided sequences to EpiFlu database: Centers for Disease Control and Prevention (USA), Chinese Center of Disease Prevention and Control, Hospital Clinic of Barcelona, National Institute of Hygiene of Morocco, National Institute of Infectious Diseases (Japan), National Institute for Medical Research (UK), Norwegian Institute of Public Health, Swedish Institute for Infectious Disease Control, Victorian Infectious Diseases Reference Laboratory (Australia).

%%% AUTHOR CONTRIBUTIONS %%%
\subsection*{Author contributions} 

AR, MAS, TB and PL conceived the study.
TB, GD, VG, JWM, AJH, CAR, DJS and AR gathered antigenic and genetic data.
AR, MAS, TB and PL designed statistical phylogenetic procedures.
TB performed the analysis.
TB, MAS, AR, JWM, AJH wrote the paper.

%%% FUNDING %%%
\subsection*{Funding} 

TB is supported by the Royal Society. 
MAS is supported by National Institutes of Health grants R01 GM086887 and R01 HG006139 and National Science Foundation grant DMS0856099.
The research leading to these results has received funding from the European Research Council under the European Community's Seventh Framework Programme (FP7/2007-2013) under Grant Agreement no. 278433-PREDEMICS and ERC Grant agreement no. 260864.
VG, AJH and JWM are funded by MRC reference no U117512723.
AR acknowledges the support of the Wellcome Trust (grant no. 092807).
Collaboration between MAS, AR and PL was supported by the National Evolutionary Synthesis Center (NESCent), NSF EF-0423641.
DJS and CAR acknowledge EU FP7 programs EMPERIE (223498) and ANTIGONE (278976), Human Frontier Science Program (HFSP) program grant
P0050/2008, Wellcome 087982AIA, and NIH Director's Pioneer Award DP1-OD000490-01.
CAR was supported by a University Research Fellowship from the Royal Society.

%%% REFERENCES %%%
\bibliographystyle{plos}
\bibliography{flux}

\pagebreak

\setcounter{figure}{0}
\setcounter{table}{0}
%\setcounter{page}{1}
\renewcommand{\thefigure}{S\arabic{figure}}
\renewcommand{\thetable}{S\arabic{table}}
%\renewcommand{\thepage}{S\arabic{page}}

%%% SUPPORTING INFORMATION %%%
\section*{Supporting Information}

%%% error_by_distance_and_year %%%
\begin{figure}[H]
	\centering		
	\includegraphics[width=0.95\textwidth]{figures/error_by_distance_and_year}
	\caption{\textbf{Absolute error in log$_2$ HI titer against antigenic distance and date of virus sampling in influenza A/H3N2.} 
	(A) Points show the absolute deviation from the predicted log$_2$ HI titer of the antigenic model, computed from virus location $\virus_i$, serum location $\serum_j$, virus effect $\ve_i$ and serum effect $\se_j$, based upon antigenic distance between virus and serum $\delta_{ij}$.
	The red line shows a LOESS regression curve of these points.
	(B) Points show the absolute deviation from the predicted log$_2$ HI titers of the antigenic model, based upon the date of virus sampling.
	The red line shows average deviations for each year.
	Relationships for A/H1N1, B/Vic and B/Yam are similar.
	} 
	\label{error_by_distance_and_year} 
\end{figure}

%%% incidence %%%
\begin{figure}[H]
	\centering		
	\includegraphics[width=0.95\textwidth]{figures/incidence}
	\caption{\textbf{Relationship between antigenic drift within and between clades and seasonal USA incidence for years 1998 to 2009.} 
	(A) Standardized incidence within a clade compared to antigenic drift within the same clade.
	For example, incidence of A/H3N2 in season 1998/1999 is measured against year-to-year antigenic drift of A/H3N2 from 1997 to 1998.
	(B) Standardized incidence within a clade compared to antigenic drift in other clades.
	For example, incidence of A/H3N2 in season 1998/1999 is measured against year-to-year antigenic drift of A/H1N1 from 1997 to 1998, as well as antigenic drift of B/Vic from 1997 to 1998.
	Standardized incidence is measured as yearly ILI percentage multiplied by proportion of viruses identified in a particular clade, and then standardized so that mean and variance across seasons and clades is equal to 0 and 1 respectively.
	Colors represent the clade in which incidence is measured; light blue for A/H3N2, purple for A/H1N1, orange for B/Vic and red for B/Yam.
	} 
	\label{incidence} 
\end{figure}

\end{document}
